
\documentclass{article}

\usepackage{amsmath}
\usepackage{amssymb}

\usepackage{parskip}

\newcommand{\EV}[3]{\ensuremath{{\cal{E}}^{#1}_{[#2 \Rightarrow #3]}}}
\newcommand{\FMAP}{\ensuremath{\langle \$ \rangle}}
\newcommand{\APPLY}{\ensuremath{\langle \star \rangle}}
\newcommand{\BIND}{\mathbin{>\!\!\!>\mkern-6.7mu=}}
\newcommand{\COMP}{\lll} %{\mathbin{<\!\!\!<\!\!\!<}}
\newcommand{\SEQ}{\ggg} %{\mathbin{>\!\!\!>\!\!\!>}}
\newcommand{\SPLIT}{\mathbin{\divideontimes\!\!\!\!\divideontimes\!\!\!\!\divideontimes}}
\newcommand{\FANOUT}{\mathbin{\&\!\!\!\&\!\!\!\&}}

\title{Algebras of Events}

\author{Frederic Peschanski\\LIP6 - Sorbonne University - CNRS UMR 7606}

\date{\today}

\begin{document}
\maketitle

\section{Deterministic events}

The type of basic non-deterministic events is denoted by: $\EV{m}{\alpha}{\beta}$

where :

\begin{itemize}
\item $m$ is a machine type
\item $\alpha$ is an input type
\item $\beta$ is an output type
\end{itemize}

\subsection{Functors}

Functor deal with the input and output of events, there is no direct interference with the machine state
 changes.

\subsubsection*{Covariant functor}

$map: (\alpha \rightarrow \beta) \rightarrow  \EV{m}{\gamma}{\alpha} \rightarrow \EV{m}{\gamma}{\beta}$

\textbf{Remark}: the expression $map f \EV{m}{\gamma}{\alpha}$ is commonly denoted by $f \FMAP \EV{m}{\gamma}{\alpha}$ 

\textbf{Usage}: transform the \underline{output} of an event with a unary function $g : \alpha \rightarrow \beta$

\subsubsection*{Contravariant functor}

$contramap: (\beta \rightarrow \alpha) \rightarrow  \EV{m}{\alpha}{\gamma} \rightarrow \EV{m}{\beta}{\gamma}$

\textbf{Usage} : transform the \underline{input} of an event with a unary function $f : \beta \rightarrow \alpha$

\subsubsection*{Profunctor}

$dimap: (\beta \rightarrow \alpha) \rightarrow (\gamma \rightarrow \delta) \rightarrow  \EV{m}{\alpha}{\gamma} \rightarrow \EV{m}{\beta}{\delta}$

\textbf{Usage}: transform both the \underline{input} and the \underline{output} of an event with :
\begin{itemize}
\item an input transformation $f : \beta \rightarrow \alpha$
\item an output transformation $g : \gamma \rightarrow \delta$
\end{itemize}

\textbf{Strong profunctor} adds:

$first' : \EV{m}{\alpha}{\beta} \rightarrow \EV{m}{\alpha \times \gamma}{\beta \times \gamma}$

and $second' : \EV{m}{\alpha}{\beta} \rightarrow \EV{m}{\gamma \times \alpha}{\gamma \times \beta}$

which allows to combine input and output transformations in tuples, and opens up the door for \textbf{lenses}.

\subsection{Applicative functors}

Applicative functors are (covariant) functors with two added requirements:

$pure: \alpha \rightarrow \EV{m}{\gamma}{\alpha}$  (for an arbirary $\gamma$)

$apply: \EV{m}{\gamma}{\alpha \rightarrow \beta} \rightarrow \EV{m}{\gamma}{\alpha} \rightarrow \EV{m}{\gamma}{\beta}$


The $apply$ operator does two things:

\begin{itemize}
\item change the state of $m$ according to the sequential composition of the effects of $ev_f$ and those of $ef_x$  (in that order)
\item apply the function $f : \alpha \rightarrow \beta$ on output value $x : \alpha$
\end{itemize}

\textbf{Remarks}:
\begin{enumerate}
\item $apply~ev_f~ev_x$ is commonly denoted by $ev_f~\APPLY~ev_x$.
\item the function $f$ is only concerned with the (covariant) output of events
\end{enumerate}

\textbf{Usage}: 
\begin{itemize}
\item $pure$ reifies values as events, but without any observable state change, i.e. a pure event is just a value.
\item $apply$ reifies function application at the functor (i.e. event) level. For instance, a standard application $f e_1 e_2 \ldots e_n$ becomes $f \FMAP e_1 \APPLY e_2 \APPLY \cdots e_n$ in the context of the functor. 
\end{itemize}

\subsection{Monads}

A monad is an applicative functor (hence also a covariant functor) that, in addition, provides the following operator :

$bind: \EV{m}{\gamma}{\alpha} \rightarrow (\alpha \rightarrow \EV{m}{\gamma}{\beta}) \rightarrow \EV{m}{\gamma}{\beta}$

\textbf{Remark}: $bind~ev~f$ is commonly denoted by $ev \BIND f$

\subsection{Categories and Arrows}

Arrows are a generalization of monads.

\subsubsection{Category}

The starting point is a \textbf{Category} (from Category Theory) with two basic operators:

$id : \EV{m}{\alpha}{\alpha}$

$comp : \EV{m}{\beta}{\gamma} \rightarrow \EV{m}{\alpha}{\beta} \rightarrow \EV{m}{\alpha}{\gamma}$

\textbf{Remarks}:  
\begin{itemize}
\item $comp~ev_2~ev_1$ is commonly denoted by $ev_2 \COMP ev_1$. 
\item this is identical to the reverse composition $ev_1 \SEQ ev_2$.
\end{itemize}

\textbf{Usage}:
\begin{itemize}
\item $id$ is an event that propates its input to the output, which is isomorphic to an Event-B event ``without'' an actual (i.e. changed) output.
\item $comp$ (and in particular the operator $\SEQ$) realizes the \textbf{sequential composition} of events, and is thus most useful
\end{itemize}

\subsubsection{Arrow}

An arrow is a category that provides two basic operators :

$arrow : (\alpha \rightarrow \beta) \rightarrow \EV{m}{\alpha}{\beta}$

$split : \EV{m}{\alpha}{\beta} \rightarrow \EV{m}{\alpha'}{\beta'} \rightarrow \EV{m}{\alpha \times \alpha'}{\beta \times \beta'}$

\textbf{Remark}: $split~ev~ev'$ is commonly denoted by $ev \SPLIT ev'$.

\textbf{Usage}:
\begin{itemize}
\item While $pure$ reifies values, $arrow$ reifies whole functions. Put in other terms, (total) functions \emph{are} (a kind of) events. For example, to connect two ``incompatible'' events $ev_1 : \EV{m}{\alpha}{\beta}$ and $ev_2 : EV{m}{\gamma}{\delta}$ in sequence, it ``suffices'' to have a ``glue'' function $g : \beta \rightarrow \gamma$ and write $ev_1 \SEQ arrow~g \SEQ ev_2$.
\item $split$ is a kind of parallel composition of events with separate and non-interfering input/output pairs.
\end{itemize}

\textbf{Derived operators}

The follwing operators are derivable from $arrow$ and $split$ :

$first : \EV{m}{\alpha}{\beta} \rightarrow \EV{m}{\alpha \times \gamma}{\beta \times \gamma}$ (for an arbitrary $\gamma$)

$second : \EV{m}{\alpha}{\beta} \rightarrow \EV{m}{\gamma \times \alpha}{\gamma \times \beta}$ (for an arbitrary $\gamma$)

$fanout : \EV{m}{\alpha}{\beta} \rightarrow \EV{m}{\alpha}{\beta'} \rightarrow \EV{m}{\alpha}{\beta \times \beta'}$

\textbf{Remarks} : 
\begin{itemize}
\item $split$ is also derivable from $first$ using the function $swap (x, y) = (y, x)$, and the following definition: 

$split = first~f \SEQ arrow~swap \SEQ first~g \SEQ arrow~swap$.
\item $fanout~ev_1~ev_2$ is commonly denoted by: $ev_1 \FANOUT ev_2$.
\end{itemize}

\textbf{Usage}:
\begin{itemize}
\item $first$ and $second$ are the two ``halves'' of $split$ 
\item $fanout$ merges to distinct events with compatible input types into a single event with a pair of outputs. Note that the effects of the two events are combined sequentially.
\end{itemize}

\end{document}
